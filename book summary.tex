\documentclass[12pt]{article}
\usepackage{etex}
\reserveinserts{28}
\usepackage{mathptmx}
\usepackage{amssymb}
\usepackage{amsmath}
\usepackage{graphicx}
\usepackage{setspace}
\usepackage{caption}
\usepackage{fullpage}
\usepackage{color}
\usepackage{pstricks}
\usepackage{rotating}
\usepackage{appendix}
\usepackage[margin=.9in]{geometry}
%\usepackage[osf]{libertine}
\usepackage{multirow}
\usepackage{xr-hyper}
\usepackage[hang,flushmargin]{footmisc}
\usepackage[breaklinks]{hyperref}
\usepackage{xr}
\externaldocument{appendices}
\usepackage{color, colortbl}
\definecolor{Gray}{gray}{0.9}
\usepackage{chronosys}
\usepackage{lscape}
\usepackage{titlesec}
\usepackage{subfig}
\usepackage{multirow}
\usepackage{array}
\usepackage{lscape}
\usepackage{natbib}
\usepackage{rotating}
 % Palladio needs more leading (space between lines)
\usepackage[T1]{fontenc}
\usepackage{pdfpages}
\usepackage[capposition=top]{floatrow}
\usepackage{array}
\usepackage{fancyhdr}


\titleformat{\subsubsection}{\itshape }{\thesubsubsection.}{0.05em}{\itshape }

\usepackage[flushleft,para]{threeparttable}
% \setlength{\parskip}{.5em}

\usepackage{tikz}
\usetikzlibrary{shapes,decorations,arrows,calc,arrows.meta,fit,positioning}
\tikzset{
 -Latex,auto,node distance =1 cm and 1 cm,semithick,
 state/.style ={ellipse, draw, minimum width = 0.7 cm},
 point/.style = {circle, draw, inner sep=0.04cm,fill,node contents={}},
 bidirected/.style={Latex-Latex,dashed},
 el/.style = {inner sep=2pt, align=left, sloped}
}

 \newcolumntype{P}[1]{>{\raggedright\arraybackslash}m{#1}}
\newcolumntype{C}[1]{>{\centering\arraybackslash}m{#1}}






%MAKE CLEAR CONTRIBUTION TO US LIT TOO

\title{ { Housing as welfare} \\ How subsidized homeownership creates social mobility through wealth, voice, and dignity in India}
\author{Tanu Kumar}

\begin{document}
\maketitle



	Two types of buildings are ubiquitous and well-known by anyone who has spent time in Mumbai. There are the ones constructed by the Slum Rehabilitation Authority: highrise apartments for those who once lived in self-constructed settlements, easily identified by the red and blue logo of six triangles stacked in a pyramid. There are also apartments constructed by the Maharashtra Housing and Area Development Authority, emblazoned with images of an orange house with a pointed roof (or an upward pointing arrow, depending on your perspective).
	These apartments have been sold for a fraction of their value through an annual lottery. Such policies exist across countries including, but not limited to, India, Brazil, Uruguay, Nigeria, Kenya, Ethiopia, and South Africa. They are particularly common in India and are in every major city, including Delhi, Mumbai, Bengaluru, Kolkata, Chennai, Hyderabad, and Ahmedabad.  How do programs that subsidize homeownership shape the lives, behavior, and futures of these citizens?
	

	 	Governments can help their citizens find shelter in a number of different ways, but there is something unique about helping households purchase a large, durable asset that they can use, rent, or sell as they wish. The support housing provides to families is in-kind, but when it is owned and can easily be bought or sold, its benefits approach those of pure income transfers. The widespread implementation of subsidized homeownership, moreover, suggests that transfers made through housing may be more politically expedient than something like a basic income guarantee. 
	 

	


\subsubsection*{Theory and contribution}

Existing research on housing policy in India and other developing countries typically evaluates their potential to address problems related to housing shortages in cities, particularly those faced by those living in informal squatter settlements. Well-studied policies in the United States and other OECD countries, on the other hand, focus on subsidizing rent for the poor. Yet governments in India and other low- and middle-income countries are actively involved in subsidizing homeownership for the poor, and we know little about the effects of this policy. 
   
I argue that subsidized homeownership generates social mobility by facilitating the accumulation of wealth among beneficiaries, and by helping and motivating them to participate to protect this newfound wealth. In this way, subsidized housing programs exhibit many of the policy feedback effects common to welfare programs in the West. But subsidized homeownership goes one step further by endowing its beneficiaries with a sense of dignity, which is an important outcome in hierarchical societies wherein most citizens are poor. Dignity for citizens may be as normatively desirable as freedom \citep{sen1999development}, but it also deepens and reinforces the effects on economic and political behavior.
 
  These three sets of effects go well beyond solving problems related to informality or a shortage of housing. The relevant programs may address problems related to the shortfall of housing, but their most important effects are through the transfer of a large and durable asset to citizens. While some may see beneficiaries' actions to resell or trade this asset on the open market as a bug that undermines a program's ability to address the lack of affordable housing for the poor, I see this flexibility as a feature that unlocks its most important gains to household welfare. 
  
  \subsubsection*{Methodology}

I illustrate the argument through a study of three housing programs in India. First, I study \textit{Indira Awas Yojana} (IAY), a  subsidy program for landless laborers and freed bonded workers that started in the late 1980s. Beneficiaries receive large grants from the government to build their own houses. I study the effects of this program through the 1994-2012 three-round India Human Development Survey \citep{desai_national_2016} panel survey. I construct exact matches of respondents who did and did not benefit and track their outcomes over three waves across 18 years. 



Second, I use a natural experiment to study a program implemented by the Maharashtra Housing and Development Authority (MHADA) in Mumbai, whereinapartments are constructed by government and sold at an extremely discounted rate to eligible low income households. The program is oversubscribed, so beneficiaries are selected through a lottery process. Importantly, beneficiaries do not have to live in the housing, but can rent it out. They may also legally sell it 10 years after winning.  I study its effects through qualitative interviews and an original 800 household survey of a sample of winners and non-winners  in 2017, conducted 3-5 years after lottery wins in 2012 and 2014. Due to the randomization, the differences in outcomes across winners and non-winners is the causally identified effect of becoming a program beneficiary.

Third, I study a program implemented by Mumbai's Slum Rehabilitation Authority (SRA).  Households living in eligible informal settlements can elect to move into government constructed apartments at no cost. If they choose to do so, the settlement is destroyed to accommodate the new housing or some other development project. Unlike the MHADA program, this program requires some form of relocation. Even if households were to rent or sell these apartments, they would not be able to return to their original housing because it is usually no longer exists. This is an important feature of the program that limits its potential. I study its effects using 40 paired qualitative interviews in 2022 of citizens living in informal settlements and citizens who chose to relocate. 


Together, these three policies illustrate the scope conditions for the argument. I show that both rural and urban programs can generate effects on wealth. Yet urban programs are likely to have stronger effects on wealth and especially voice. This is because urban housing is more subject to increases in market value due to an excess of demand common in these contexts. Within urban areas, the comparison between the SRA and MHADA programs suggests that policies are likely to generate the largest effects when they do not require relocation. This finding is complementary to research finding that relocation can undermine the gains to housing programs due to broken social networks and distance from labor markets \citep{barnhardt_moving_2017, picarelli_there_2019}.


The mixed-methods design also aims to reveal multiple dimensions of policy effects. The panel study of IAY allows me to show how a policy may lead to divergences in economic prospects over time. The natural experiment used to measure the effects of the MHADA program addresses problems of selection that typically plague policy evaluations.  The qualitative surveys, finally, allow me to understand and describe how a policy shapes citizens' lives in their own words. 


\bibliography{lib.bib}

\bibliographystyle{apalike}


\end{document}