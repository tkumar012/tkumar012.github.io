% Autogenerated translation of 04_Book.md by Texpad
% To stop this file being overwritten during the typeset process, please move or remove this header

\documentclass[12pt]{book}
\usepackage{graphicx}
\usepackage{fontspec}
\usepackage[utf8]{inputenc}
\usepackage[a4paper,left=.5in,right=.5in,top=.3in,bottom=0.3in]{geometry}
\setlength\parindent{0pt}
\setlength{\parskip}{\baselineskip}
\setmainfont{Helvetica Neue}
\usepackage{hyperref}
\pagestyle{plain}
\begin{document}

\hrule
layout: page
title: Book
permalink: /Book/

\section*{order: 4}

\begin{verbatim}
<!-- Global site tag (gtag.js) - Google Analytics -->
\end{verbatim}

\begin{verbatim}
<script async src="https://www.googletagmanager.com/gtag/js?id=UA-111923831-1"></script>
<script>
  window.dataLayer = window.dataLayer || [];
  function gtag(){dataLayer.push(arguments);}
  gtag('js', new Date());

  gtag('config', 'UA-111923831-1');
</script>

\end{verbatim}

\section*{Housing as Welfare}

\subsection*{How subsidized homeownership generates social mobility through wealth, voice, and dignity in India}

 How do programs that subsidize homeownership shape the lives, behavior, and futures of these citizens? Generally, rural housing programs in India are criticized for providing citizens with poor quality shelters that do not meet their needs. Urban public housing programs suffer from the same problem, and also relocate beneficiaries to greenfield stigmatized development sites that are close to health hazards and far from work, family, and friends. These issues are not so different from the ones that plague large public housing projects in the United States, like Chicago's Cabrini-Green homes or Pruitt-Igoe public housing in St. Louis.

  In this book, I reimagine this type of housing program as one whose primary function is not to provide shelter or relocate households, but to transfer a large asset to citizens. The support housing provides to families is in-kind, but when it is owned and can easily be bought or sold, its benefits approach those of pure income transfers; in fact, housing goes beyond income and provides a textit\{wealth\} transfer. Households use these transfers to invest in their futures and build intergenerational wealth. 

These asset transfers further change citizens' sense of dignity. The flexibility and certainty with which citizens can consume these transfers gives them a degree of agency and control over their lives. It also changes how the see their relationships with others, especially those who may be higher in traditional structures of power. 

Wealth and dignity, in turn, fundamentally shape how citizens participate in the broader political arena. They are emboldened  to make demands of their local government. They are, furthermore,  especially motivated to do so as they seek to advance their own interests and protect the value of their new homes. 	

Based on studies of three housing policies, administrative data, household surveys, a natural experiment, and in-depth qualitative interviews,  I argue that subsidized homeownership helps even the poorest households build wealth, live with dignity, and exercise their voice as citizens--in both rural and urban areas.  

\emph{Manuscript in progress, chapters available upon request}

\end{document}
