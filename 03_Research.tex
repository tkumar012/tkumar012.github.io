% Autogenerated translation of 03_Research.md by Texpad
% To stop this file being overwritten during the typeset process, please move or remove this header

\documentclass[12pt]{book}
\usepackage{graphicx}
\usepackage{fontspec}
\usepackage[utf8]{inputenc}
\usepackage[a4paper,left=.5in,right=.5in,top=.3in,bottom=0.3in]{geometry}
\setlength\parindent{0pt}
\setlength{\parskip}{\baselineskip}
\setmainfont{Helvetica Neue}
\usepackage{hyperref}
\pagestyle{plain}
\begin{document}

\hrule
layout: page
title: Research
permalink: /Research/

\section*{order: 3}

\section*{Peer-reviewed Publications}

\textbf{Kumar, T., Post, A., and Ray, I. "Flows, Leaks, and Blockages in Informational Interventions: A Field Experimental Study of Bangalore's Water Sector." \emph{World Development} 106: 149-160, 2018.} (\href{https://docs.google.com/viewer?a=v&pid=sites&srcid=ZGVmYXVsdGRvbWFpbnxhbGlzb25lcG9zdHxneDo2MjRlMWRiZDNlYzJlNWRl}{Paper} and \href{https://dataverse.harvard.edu/dataset.xhtml?persistentId=doi:10.7910/DVN/ZMYDWN}{data})

 \emph{Abstract}: Many policies and programs based on informational interventions hinge upon the assumption that providing citizens with information can help improve the quality of public services, or help citizens cope with poor services. We present a causal framework that can be used to identify leaks and blockages in the information production and dissemination process in such programs. We conceptualize the “information pipeline” as a series of connected nodes, each of which constitutes a possible point of blockage. We apply the framework to a field-experimental evaluation of a program that provided households in Bangalore, India, with advance notification of intermittently provided piped water. Our study detected no impacts on household wait times for water or on how citizens viewed the state, but found that notifications reduced stress. Our framework reveals that, in our case, noncompliance among human intermediaries and asymmetric gender relations contributed in large part to these null-to-modest results. Diagnostic frameworks like this should be used more extensively in development research to better understand the mechanisms responsible for program success and failure, to identify subgroups that actually received the intended treatment, and to identify potential leaks and blockages when replicating existing programs in new settings.

\section*{Working Papers}

\textbf{"Welfare beneficiaries participate in local politics: the effects of affordable housing in Mumbai"} (\href{https://www.dropbox.com/s/k3q9838eygfpvmi/Kumar_housing2019.pdf?dl=0}{Paper})

\emph{Abstract}: How do welfare policies affect political participation among beneficiaries in developing countries? I argue that beneficiaries are motivated to take part in local politics to improve services that affect the real value of the benefits. I support the argument with a study of affordable housing, a welfare program that is common in India. I present a natural experiment in which I tracked down and surveyed 834 winners and applicants of multiple affordable housing lotteries in Mumbai. I find that winning a home increases reported voting and complaint-making to improve neighborhoods, improvements that that will increase the value of the homes. Increases in reported action also occur among those who do not move into the new homes, but rent them out instead. This reported behavior is accompanied by actual increases in knowledge about local politics. Finally, I show that in Mumbai, a city with bureaucratic systems for addressing complaints, citizen demands can successfully improve the quality of certain services. The findings suggest that through the channel of increased civic participation, social welfare policies can increase citizen agency over local government.

\textbf{"The effects of subsidized assets for sale: Evidence from an affordable housing program in Mumbai."}

\emph{Abstract}: How do affordable housing programs in developing countries affect beneficiaries? I study a policy model common to urban India wherein homes are sold, rather than rented, at subsidized prices. Beneficiaries receive a large wealth transfer and an asset that either delivers a stream of housing benefits or rental income. As in many other settings, homeownership here can also serve as a powerful symbol of upward mobility. Through surveys of winners and applicants of multiple housing lotteries in Mumbai, I find that on average, the program leads to improvements in beneficiaries' housing quality with only small increases in distance from friends, relatives, and work. As with any asset purchase, the intervention entails a transfer of income across time through the form of mortgage payments. I accordingly observe decreases in monthly cash savings. Nevertheless, I find that that beneficiaries express lower stress about their finances and greater optimism about the future. The program also leads to increases in educational attainment, which I attribute to increased feelings of financial security. This paper is among the first to look at the effects of the subsidized sale of an asset in a developing country.

\textbf{"From public service access to service quality: The distributive politics of piped water in Bangalore" with Alison Post, Megan Otsuka, Francesc Pardo-Bosch, and Isha Ray (under review)}

\emph{Abstract}: The local public goods and distributive politics literatures focus overwhelmingly on government spending and service access. Yet service quality can vary dramatically and be targeted strategically. This paper provides one of the first analyses of the allocation of a key dimension of service quality: intermittency, which characterizes vital services like water and electricity for hundreds of millions of people. Analyzing the distribution of intermittent infrastructure services, we show, requires examining how infrastructure networks shape allocations. Our data on piped water flows in Bangalore reveals that intermittency is not correlated with household characteristics but with characteristics of "valve areas," or the smallest units to which water is distributed. Contrary to predictions from the distributive politics literature, valve areas with higher proportions of low- income and low-caste households receive more frequent and predictable service. This suggests that the politics of access to the network differ from those directing water flows within the network.

\textbf{"Attitudes towards reservations in India: Does prejudice lie beneath the rhetoric of fairness?" with Pradeep Chhibber and Rahul Verma (under review)}

\emph{Abstract}: Many have opposed affirmative action in India by arguing that quotas decrease opportunities for the general population. Yet research on attitudes towards quotas from other countries and the prevalence of caste-based discrimination in India suggests that opposition to these quotas may also stem from a distaste for sharing spaces with members of certain caste groups. We use a survey experiment to determine whether support for quotas is more susceptible to appeals about the scarcity of opportunities or those that elicit feelings of prejudice for particular groups. We find that the effects of these appeals varies with the social "rank" of group in question.

\end{document}
